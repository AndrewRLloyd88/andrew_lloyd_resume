\documentclass[11pt,a4paper, sans]{moderncv}
\usepackage{fontspec}
\usepackage{fontawesome}
\moderncvtheme[blue]{casual}
\usepackage[utf8]{inputenc}
\usepackage[scale=0.9]{geometry}                      
\usepackage{multicol}



% personal data
\firstname{Andrew}
\familyname{Lloyd}
\title{Full Stack Web Developer}          % optional, remove / comment the line if not wanted
\mobile{778 679 8485}                     % optional, remove / comment the line if not wanted
\email{andrew.lloyd01@googlemail.com}  
\homepage{www.arlmedia.ca}          % optional, 
\address{}{}{Victoria, BC}          % optional, 

\begin{document}
%-----       resume       ---------------------------------------------------------
\makecvtitle

\section{Summary} % (fold)
\label{sec:summary}

From an early age I was introduced to computers by my brother in law, and ever since I have always had a fascination with software development coupled with a curiosity about building things and problem solving. Working in the film industry and as a lecturer formerly, programming was a hobby that kept on drawing me back in. This year I was laid off from my role as an office administrator, which spurred me on to go back to school and pursue my dream of web development. Not only have do I find coding intellectually stimulating but I have quickly realised that I am now doing what I love. Since starting Lighthouse Labs' 12 week bootcamp I have read through more material than I thought humanly possible to, taken extra Udemy courses on parallel to my studies, and created a portfolio of applications. Working as a full stack developer has been a transformative, rewarding and fulfilling experience, that has helped me realise my life long dream.


\section{Skills}
\begin{itemize}
  \begin{multicols}{3}
  \item JavaScript
  \item HTML5
  \item CSS
  \item Node/Express
  \item C$\sharp$ 
  \item React
  \item Storybook
  \item Cypress
  \item Jest
  \item Ruby
  \item AJAX
  \item NPM
  \item SQL
  \item Adobe Creative Suite
  \item Avid Pro Tools
  \item Unity3D
\end{multicols}
\end{itemize}
%Section Projects

\section{Projects}

\cventry{Sept. 26 - Oct. 7, 2020}{Interview Scheduler \emph{}}{Lighthouse Labs - Victoria, BC}{}{}{} % arguments 3 to 6 can be left empty

\cventry{March-May 2015}{Visiting student -- Software Practices Lab}{University of British Columbia, Vancouver, Canada}{}{}{Developer activity and software quality in open-source\\
Advisors: Dr. Marc Palyart and Pr. Gail C. Murphy}

\cventry{2012}{M.Sc. in Computer Science}{University of Bordeaux, Bordeaux, France}{}{}{Software Engineering - Project Management}


\section{Professional Experience}
\cventry{April - September 2012}{CEA \emph{(French Atomic Energy and Alternative Energies
Commission)}}{R\&D Engineer Intern}{}{}{Designed an approach to adapt an existing version control system (Git) to a domain-specific modelling language.}
\cventry{April - July 2010}{Keyland IT, Burgos, Spain}{Intern Developer}{}{}{Developed a set of modules for a civil engineering project management system.}

\section{Education}

\cventry{2019}{Diploma \emph{}}{Lighthouse Labs - Victoria, BC}{}{}{} % arguments 3 to 6 can be left empty

\cventry{March-May 2015}{Visiting student -- Software Practices Lab}{University of British Columbia, Vancouver, Canada}{}{}{Developer activity and software quality in open-source\\
Advisors: Dr. Marc Palyart and Pr. Gail C. Murphy}

\cventry{2012}{M.Sc. in Computer Science}{University of Bordeaux, Bordeaux, France}{}{}{Software Engineering - Project Management}


\section{Professional Experience}
\cventry{April - September 2012}{CEA \emph{(French Atomic Energy and Alternative Energies
Commission)}}{R\&D Engineer Intern}{}{}{Designed an approach to adapt an existing version control system (Git) to a domain-specific modelling language.}
\cventry{April - July 2010}{Keyland IT, Burgos, Spain}{Intern Developer}{}{}{Developed a set of modules for a civil engineering project management system.}


\section{Selected Publications}

\cvitem{}{\textbf{Matthieu Foucault}, Marc Palyart, Xavier Blanc, Gail C. Murphy and Jean-Rémy Falleri.
	\emph{Impact of developer turnover on quality in open-source software.}
10th Joint Meeting of the European Software Engineering Conference and the ACM SIGSOFT Symposium on the Foundations of Software Engineering (ESEC/FSE), 2015.}

\cvitem{}{\textbf{Matthieu Foucault}, Cédric Teyton, David Lo, Xavier Blanc and Jean-Rémy Falleri.
	\emph{On the usefulness of ownership metrics in open-source software projects.}
	 Information \& Software Technology (IST), 2015.}

\cvitem{}{\textbf{Matthieu Foucault}, Jean-Rémy Falleri and Xavier Blanc.
\emph{Code ownership in open-source software.}
18th International Conference on Evaluation and Assessment in Software Engineering (EASE), 2014.}

\cvitem{}{\textbf{Matthieu Foucault}, Marc Palyart, Jean-Rémy Falleri, Xavier Blanc.
\emph{Computing contextual metric thresholds.}
29th Symposium On Applied Computing (SAC), 2014.}

% \cvitem{}{Jean-Rémy Falleri, Cédric Teyton, \textbf{Matthieu Foucault}, Marc Palyart, Floréal Morandat and Xavier Blanc\\
% \emph{The Harmony Platform.}\\
% Technical Report, (see \url{http://arxiv.org/pdf/1306.6262.pdf}), 2013.}

%\cvitem{}{\textbf{Matthieu Foucault}, Sébastien Barbier, David Lugato.
%\emph{Enhancing version control with domain-specific semantics.}
%5th International Workshop on Modeling in Software Engineering (MiSE), 2013.}

\newpage

\section{Academic Experience}
\cvitem{2015}{Teaching: Web development with the MEAN stack, University of Bordeaux}
\cvitem{2014}{Supervising one graduate student for 5 months, University of Bordeaux}
\cvitem{2012-2014}{Teaching: REST architecture and Java Servlets, University of Bordeaux}
\cvitem{August 2013}{Participation in Summer School: Empirical Research Methods in Informatics - Technical University of Denmark, Kongens Lyngby, Denmark - Teacher: Prof Dr. Harald Störrle}

\section{References}
\setlength{\hintscolumnwidth}{4cm}
	\cvitem{\textbf{Xavier Blanc}}{Professor, University of Bordeaux, France }
	\cvitem{}{\url{http://www.labri.fr/perso/xblanc/}}
	\cvitem{}{xavier.blanc@labri.fr}
	\cvitem{\textbf{Jean-Rémy Falleri}}{Associate Professor, University of Bordeaux, France}
	\cvitem{}{\url{http://www.labri.fr/perso/falleri/}}
	\cvitem{}{falleri@labri.fr}
	\cvitem{\textbf{Gail C. Murphy}}{Professor, University of British Columbia, Canada}
	\cvitem{}{\url{http://www.cs.ubc.ca/~murphy/}}
	\cvitem{}{murphy@cs.ubc.ca}
	\cvitem{\textbf{David Lo}}{Assistant Professor in School of Information Systems, Singapore Management University}
	\cvitem{}{\url{http://www.mysmu.edu/faculty/davidlo/}}
	\cvitem{}{davidlo@smu.edu.sg}

\renewcommand{\listitemsymbol}{-~}            % change the symbol for lists

% Publications from a BibTeX file without multibib
%  for numerical labels: \renewcommand{\bibliographyitemlabel}{\@biblabel{\arabic{enumiv}}}
%  to redefine the heading string ("Publications"): \renewcommand{\refname}{Articles}
%\nocite{*}
% \bibliographystyle{plain}
% \bibliography{publications}                   % 'publications' is the name of a BibTeX file

% Publications from a BibTeX file using the multibib package
%\section{Publications}
%\nocitebook{book1,book2}
%\bibliographystylebook{plain}
%\bibliographybook{publications}              % 'publications' is the name of a BibTeX file
%\nocitemisc{misc1,misc2,misc3}
%\bibliographystylemisc{plain}
%\bibliographymisc{publications}              % 'publications' is the name of a BibTeX file

\clearpage

\end{document}